\chapter{Měření odezvy \label{mereni}}

Pro měření jsem použil nástroj \emph{ab}, určený pro měření odezvy HTTP serverů \autocite{ab}. K~běhu testovaných služeb jsem použil Gunicorn, HTTP server napsaný v~Pythonu \autocite{gunicorn}.

U~měření zaměřených na autentizaci a autorizaci jsem vynechal implementaci v~sandmanu2, jelikož ta příslušné části neobsahuje. Autentizaci jsem prováděl proti falešnému OAAS z~vlastního modulu \verb!utvsapitoken!, který běžel na stejném počítači jako testované služby. U~ostatních měření jsem autentizaci i autorizaci vypnul, aby neovlivňovala měření.

Data byla při měření dostupná v~MariaDB databázi běžící na stejném počítači jako testované služby.

HTTP server byl spuštěn s~dvěma vlákny. Měřící nástroj \emph{ab} službu testoval pěti tisíci požadavky, v~dávkách po jednom stu, kde jedna dávka vždy probíhala současně.

Zde prezentované měření rychlosti odezvy bylo prováděno na konkrétních implementacích popsaných v~kapitole \emph{\nameref{implementace}}. Vzhledem k~tomu, že jednotlivé měřené implementace jsou netriviální, nelze zde prezentované výsledky v~žádném případě generalizovat na celý použitý framework.

U~jednotlivých měření uvádím příklad požadavku. Pro jednotlivé implementace se tyto požadavky mohou mírně lišit, ale pro přehlednost uvádím pouze jeden.

Záznamy z~měření s~kompletními požadavky, výstupy a naměřenými rychlostmi jsou dostupné na přiloženém médiu. Testovací skripty jsou také dostupné na přiloženém médiu a na adrese:

\url{https://github.com/hroncok/utvsapi-benchamrk}

\section{Zobrazení jedné položky}\label{zobrazenuxed-jednuxe9-poloux17eky}

Při tomto měření byl testován požadavek na jednu položku:

\verb!GET /enrollments/25563/!

Jak můžete vidět \protect\hyperlink{pic:item:chart}{z grafu}, nejrychlejší je zde implementace v~ripozu a nejpomalejší v~Django REST frameworku.

\begin{figure}
\centering
\includegraphics{pdfs/item_chart}
\caption{Rychlost: Jedna položka\label{pic:item:chart}}
\end{figure}

\section{Zobrazení seznamu položek}\label{zobrazenuxed-seznamu-poloux17eek}

Při tomto měření byl testován požadavek na jednu stránku seznamu položek o~délce dvacet:

\verb!GET /enrollments/?page_size=20!

Je třeba poznamenat, že ripozo v~seznamu uvádí jen odkazy na jednotlivé položky a ostatní frameworky serializují všech dvacet objektů. Bohužel ripozo jinou možnost nenabízí, tento test má tedy méně vypovídající hodnotu. Proto je v~grafu u~ripoza hvězdička.

Jak můžete vidět \protect\hyperlink{pic:list:chart}{z grafu}, nejrychlejší je zde nepřekvapivě právě implementace v~ripozu, nejpomalejší pak v~sandmanu2.

\begin{figure}
\centering
\includegraphics{pdfs/list_chart}
\caption{Rychlost: Seznam položek\label{pic:list:chart}}
\end{figure}

\section{Filtrování seznamu položek}\label{filtrovuxe1nuxed-seznamu-poloux17eek}

Při tomto měření byl testován požadavek na seznam kurzů, které probíhají v~pátek, a velikost byla opět omezena na dvacet:

\verb!GET /courses/?page_size=20&day=5!

Platí stejná poznámka jako u~minulého měření -- ripozo je zde ve značné výhodě. V~grafu je proto označeno hvězdičkou.

Jak můžete vidět \protect\hyperlink{pic:filter:chart}{z grafu}, nejrychlejší je opět implementace v~ripozu, ale již nemá takový náskok, nejpomalejší je opět implementace v~sandmanu2.

\begin{figure}
\centering
\includegraphics{pdfs/filter_chart}
\caption{Rychlost: Filtrovaný seznam položek\label{pic:filter:chart}}
\end{figure}

\section{Jednoduchá autorizace}\label{jednoduchuxe1-autorizace}

Při tomto měření byl testován požadavek na jednu položku z~jiného zdroje než \verb!/enrollments/!, tedy u~zdroje, kde je autorizační logika jednodušší:

\verb!GET /courses/1/                    Authorization: Bearer ...!

Jak můžete vidět \protect\hyperlink{pic:simple:auth:item:chart}{z grafu}, nejrychlejší je opět implementace v~ripozu (zde už \emph{není} zvýhodněný), nejpomalejší implementace v~Django REST frameworku.

\begin{figure}
\centering
\includegraphics{pdfs/simple_auth_item_chart}
\caption{Rychlost: Jedna položka s~jednoduchou autorizací\label{pic:simple:auth:item:chart}}
\end{figure}

\section{Komplexní autorizace}\label{komplexnuxed-autorizace}

Při tomto měření byl testován požadavek na jednu položku ze zdroje, kde je autorizační logika komplexní:

\verb!GET /enrollments/25563/            Authorization: Bearer ...!

Byly provedeny tři měření, pokaždé s~jiným druhem tokenu (studentský, zaměstnanecký a „všemocný“). Vzhledem k~tomu, že se jednotlivé výsledky lišily jen o~malou aditivní konstantu, prezentuji zde průměr z~těchto tří měření.

Jak můžete vidět \protect\hyperlink{pic:god:teacher:student:auth:item:chart}{z grafu}, nejrychlejší je opět implementace v~ripozu, nejpomalejší pak implementace v~Django REST frameworku.

\begin{figure}
\centering
\includegraphics{pdfs/god_teacher_student_auth_item_chart}
\caption{Rychlost: Jedna položka s~komplexní autorizací\label{pic:god:teacher:student:auth:item:chart}}
\end{figure}

Provedl jsem i měření pro seznam položek s~komplexní autorizací:

\verb!GET /enrollments/?page_size=20     Authorization: Bearer ...!

Zde se výsledky lišily podle toho, jestli se jednalo o~studentský token či nikoliv. \protect\hyperlink{pic:god:teacher:auth:list:chart}{V~grafu} je zobrazen průměr pro zaměstnanecký a všemocný token; nejrychlejší je implementace v~ripozu, která je zde opět ve velké výhodě, protože se jedná o~seznam položek. Implementace v~Django REST frameworku je mírně pomalejší než implementace v~Eve.

\begin{figure}
\centering
\includegraphics{pdfs/god_teacher_auth_list_chart}
\caption{Rychlost: Seznam položek s~komplexní autorizací (nestudent)\label{pic:god:teacher:auth:list:chart}}
\end{figure}

\protect\hyperlink{pic:student:auth:list:chart}{V~grafu} jsou vidět výsledky pro studentský token. Zde je ripozo pořád ve výhodě, ale přesto zaostává za nejrychlejším Django REST frameworkem i druhým Eve.

\begin{figure}
\centering
\includegraphics{pdfs/student_auth_list_chart}
\caption{Rychlost: Seznam položek s~komplexní autorizací (student)\label{pic:student:auth:list:chart}}
\end{figure}

Vzhledem k~rozdílným výsledkům pro různé druhy požadavků se zdráhám z~měření vyvozovat nějaké závěry.

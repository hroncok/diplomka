Jak říká Zen of Python \autocite{PEP20}, měl by být jeden -- a nejlépe pouze jeden -- zřejmý způsob, jak to udělat. V~případě webových RESTful API to ale v~Pythonu neplatí, existuje celá řada open-source knihoven a frameworků, které umožňují RESTful API vytvářet. Některé fungují pouze společně s~frameworky na tvorbu webových aplikací, jako například s~Flaskem nebo Djangem, jiné fungují samostatně.

V~této diplomové práci se budu zabývat téměř dvacítkou těchto frameworků. V~kapitole \emph{\nameref{frameworky}} je zhodnotím z~hlediska použitelnosti, množství nabízených funkcí, podpory standardů, množství závislostí, ale i z~hlediska stavu a oblíbenosti projektu.

Abych mohl frameworky zkoumat více do hloubky a porovnat je i z~hlediska rychlosti, navrhnu ukázkovou službu poskytující RESTful API pro přístup k~rozvrhovým datům Ústavu tělesné výchovy a sportu ČVUT v~Praze. Návrhem se budu zabývat v~kapitole \emph{\nameref{navrh}}.

Tuto službu pak implementuji ve čtyřech vybraných frameworcích na základě předchozího zkoumání a hodnocení; o~tom prozradí více kapitola \emph{\nameref{implementace}}. Hotová řešení pak podrobím testování doby odezvy v~kapitole \emph{\nameref{mereni}}.

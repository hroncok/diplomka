\section{Tastypie}\label{tastypie}

\begin{figure}
\centering
\includegraphics{images/tastypie}
\caption{Logo Tastypie \autocite{tastypie}\label{pic:tastypie}}
\end{figure}

Tastypie je framework pro vytváření API k~webovým službám určený pro Django. Poskytuje abstrakci pro vytváření rozhraní ve stylu REST. Zjednodušuje zveřejňování modelů a umožňuje vybrat, které části budou přes API přístupné. Kromě ORM dat je možné použít i jiné zdroje. \autocite{tastypie}

Mezi hlavní funkce patří \autocite{tastypie}:

\begin{itemize}
\tightlist
\item
  podpora HTTP metod GET, POST, PUT, DELETE a PATCH,
\item
  rozumné chování ve výchozím stavu,
\item
  rozšířitelnost,
\item
  podpora různých serializačních formátů (JSON/XML/YAML/bplist),
\item
  HATEOAS,
\item
  dobré testy a dokumentace.
\end{itemize}

Projekt vznikl již v~roce 2010, od té doby vyšlo více než dvacet verzí, poslední necelý měsíc před psaním tohoto textu. Autorem projektu je jednotlivec Daniel Lindsley, který se projektu již příliš nevěnuje, v~současnosti se o~něj stará Seán Hayes, přispělo celkem více než 150 přispěvatelů. Projekt je distribuován pod permisivní BSD licencí \autocite{BSD3}.

Pokud si vystačíte s~JSON serializací, závisí Tastypie přímo na třech, nepřímo na čtyřech modulech a zabírá 41~MiB. Pro použití XML, YAML nebo bplistu je potřeba nainstalovat další moduly. Tastypie funguje na Pythonu 2 i 3 a podporuje poslední verze Djanga.

\subsection{HATEOAS}\label{hateoas}

Příklad použití můžete vidět \protect\hyperlink{code:tastypie}{v~ukázkách} \protect\hyperlink{code:tastypie2}{a}. Přímo v~tomto příkladu vznikne prolinkování mezi zdroji pomocí URL. Tastypie dostává tři body.

\begin{listing}[htbp]
\caption{{\label{code:tastypie}Příklad použití z~dokumentace Tastypie (api.py) \autocite{tastypiedoc}}}
\begin{minted}[bgcolor=codebg]{python}
from django.contrib.auth.models import User
from tastypie import fields
from tastypie.resources import ModelResource
from myapp.models import Entry


class UserResource(ModelResource):
    class Meta:
        queryset = User.objects.all()
        resource_name = 'user'


class EntryResource(ModelResource):
    user = fields.ForeignKey(UserResource, 'user')

    class Meta:
        queryset = Entry.objects.all()
        resource_name = 'entry'
\end{minted}
\end{listing}

\begin{listing}[htbp]
\caption{{\label{code:tastypie2}Příklad použití z~dokumentace Tastypie (urls.py) \autocite{tastypiedoc}}}
\begin{minted}[bgcolor=codebg]{python}
from django.conf.urls import url, include
from tastypie.api import Api
from myapp.api import EntryResource, UserResource

v1_api = Api(api_name='v1')
v1_api.register(UserResource())
v1_api.register(EntryResource())

urlpatterns = [
    # The normal jazz here...
    url(r'^blog/', include('myapp.urls')),
    url(r'^api/', include(v1_api.urls)),
]
\end{minted}
\end{listing}

\subsection{Přístupová práva}\label{pux159uxedstupovuxe1-pruxe1va}

Tastypie umožňuje autentizaci přes HTTP jméno a heslo, pomocí API klíče, session a OAuth 1; lze si také dopsat vlastní způsob \autocite{tastypieauth}. Existují moduly třetích stran přidávající podporu OAuth 2 \autocite{tastypieoath}. Na úrovni zdrojů lze pak nastavit, jaký autorizační model se použije, k~dispozici je buďto varianta povolit všechno, nebo povolit jen číst, případně lze použít propracovanější systém Djanga, který mapuje práva uživatele na konkrétní objekty; implementace vlastní logiky je také možná \autocite{tastypieauto}. I~zde tedy Tastypie získává tři body.

Tastypie se jeví jako velmi použitelný framework pro Django. Důstojně konkuruje Django REST frameworku, o~kterém jsem psal \protect\hyperlink{drf:fra}{v~části}. Případná volba mezi těmito dvěma frameworky hodně závisí na konkrétních potřebách a preferencích uživatele.

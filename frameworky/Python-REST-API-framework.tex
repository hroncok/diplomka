\section{Python REST API framework}\label{python-rest-api-framework}

Python REST API framework (zkráceně PRAF) je sada nástrojů postavená na Werkzeugu, pro snadnou tvorbu RESTful API pomocí MVC architektury \autocite{praf}. Mezi hlavní funkce patří \autocite{praf}:

\begin{itemize}
\tightlist
\item
  stránkování,
\item
  autentizace,
\item
  autorizace,
\item
  filtry,
\item
  částečné odpovědi,
\item
  řízení chyb,
\item
  validátory dat,
\item
  formátovače dat.
\end{itemize}

PRAF obsahuje několik součástí, které je potřeba využít k~tvorbě API \autocite{prafarch}:

\begin{itemize}
\tightlist
\item
  \textbf{datastore} je třída, která nějakým způsobem obstarává data, implicitně může využít buďto SQLite nebo reprezentaci v~paměti, pro cokoli jiného musíte implementovat vlastní třídu podle daného rozhraní;
\item
  \textbf{modely} slouží k~popsání jednotlivých typů dat v~\emph{datastore};
\item
  \textbf{controller} obsluhuje jeden resource, ve kterém se přistupuje k~datům z~jednoho modelu v~\emph{datastore};
\item
  \textbf{pohledy} pak definují, jakým způsobem budou data prezentována.
\end{itemize}

Konkrétní příklad můžete vidět \protect\hyperlink{code:praf}{v~ukázce}. Dokumentace obsahuje také komplexnější příklad včetně obsáhlého tutoriálu, jak jej vytvořit \autocite{praftuto}. Kromě tutoriálu je však dokumentace velmi stručná a místy se zdá, že možnosti PRAF příliš nepřesahují rozsah uvedeného příkladu.

\begin{listing}[htbp]
\caption{{\label{code:praf}Příklad použití z~dokumentace PRAF \autocite{praf}}}
\begin{minted}[bgcolor=codebg]{python}
from rest_api_framework import models
from rest_api_framework.datastore import SQLiteDataStore
from rest_api_framework.views import JsonResponse
from rest_api_framework.controllers import Controller
from rest_api_framework.datastore.validators import UniqueTogether
from rest_api_framework.pagination import Pagination


class UserModel(models.Model):
    """Define how to handle and validate your data."""
    fields = [models.StringField(name="first_name", required=True),
              models.StringField(name="last_name", required=True),
              models.PkField(name="id", required=True)
              ]


def remove_id(response, obj):
    """Do not show the id in the response."""
    obj.pop(response.model.pk_field.name)
    return obj


class UserEndPoint(Controller):
    ressource = {
        "ressource_name": "users",
        "ressource": {"name": "adress_book.db", "table": "users"},
        "model": UserModel,
        "datastore": SQLiteDataStore,
        "options": {"validators": [UniqueTogether("first_name",
                                                  "last_name")]}
        }

    controller = {
        "list_verbs": ["GET", "POST"],
        "unique_verbs": ["GET", "PUT", "DELETE"],
        "options": {"pagination": Pagination(20)}
        }

    view = {"response_class": JsonResponse,
            "options": {"formaters": ["add_ressource_uri",
                                      remove_id]}}


if __name__ == '__main__':
    from werkzeug.serving import run_simple
    from rest_api_framework.controllers import WSGIDispatcher
    app = WSGIDispatcher([UserEndPoint])
    run_simple('127.0.0.1', 5000, app, use_debugger=True,
               use_reloader=True)
\end{minted}
\end{listing}

Python REST API framework je distribuován pod MIT licencí \autocite{MIT}. Za projektem stojí jednotlivec Yohann Gabory, s~minimálním přispěním od dalších vývojářů. PRAF vznikl v~roce 2013 a od té doby vyšly pouze čtyři verze. Vývoj není příliš aktivní, v~posledních dvou letech přibylo jen několik jednotek commitů.

Instalace přímo závisí na dvou knihovnách včetně Werkzeugu, celkem pak nepřímo na třech. Včetně závislostí má 15~988 řádků kódu. Na GitHubu má projekt pouze čtyři hvězdy a z~PyPI byl za poslední měsíc stažen jen o~něco málo více než dvousetkrát. Projekt neobsahuje informaci o~tom, jestli podporuje Python 3, ale obsahuje minimálně jeden řádek kódu napsaný v~nekompatibilní syntaxi, z~čehož soudím, že Python 3 nepodporuje.

\subsection{HATEOAS}\label{hateoas}

Dokumentace v~tutoriálu uvádí postup, jak prolinkovat jednotlivé zdroje mezi sebou \autocite{praflink1}\autocite{praflink2}. Framework sám tuto funkci neobsahuje, ale ukazuje příklad formátovače, který je znovupoužitelný v~celém projektu a zajistí, aby všechny cizí klíče byly reprezentovány pomocí odkazu. Můžete jej vidět \protect\hyperlink{code:praflink}{v~ukázce}. Za možnost nějak prolinkovat zdroje dostává Python REST API framework jeden bod.

\begin{listing}[htbp]
\caption{{\label{code:praflink}PRAF: Formátovač pro prolinkování dat \autocite{praflink2}}}
\begin{minted}[bgcolor=codebg]{python}
def format_foreign_key(response, obj):
    from rest_api_framework.models.fields import ForeignKeyField
    for f in response.model.get_fields():
        if isinstance(f, ForeignKeyField):
            obj[f.name] = \
                "/{0}/{1}/".format(f.options["foreign"]["table"],
                                   obj[f.name])
    return obj
\end{minted}
\end{listing}

\subsection{Přístupová práva}\label{pux159uxedstupovuxe1-pruxe1va}

Tutoriál opět uvádí postup \autocite{prafauth}, jak implementovat autentizaci, v~tomto konkrétním případě API klíčem předaným pomocí GET parametru zakódovaným v~URL. Pokud chcete, můžete si samozřejmě implementovat způsob vlastní.

V~případě autorizace nabízí PRAF pouze možnost zpřístupnit daný zdroj všem autentizovaným požadavkům \autocite{prafauth}, implementace komplexnějších přístupových práv je opět možná. Proto dávám dva body.

Python REST API framework nabízí určitou strukturu, jak REST API v~Pythonu budovat, nenabízí ale velký výběr stavebních kamenů. Předpokládá se, že programátor si je dobuduje sám, což nepovažuji nutně za špatnou věc. Je však třeba vytknout v~současnosti zpomalený vývoj projektu a především absenci podpory pro Python 3. Nízká oblíbenost projektu může být důsledkem těchto problémů.

\section{Cornice}\label{cornice}

Cornice je RESTový framework pro Pyramid \autocite{cornice}. Je vyvíjen lidmi z~Mozilla Services a vydán pod Mozilla Public License 2.0 \autocite{mpl2}, čímž se řadí do kategorie LGPL. Závisí na dvou dalších modulech (\verb!pyramid! a \verb!simplejson!), čímž nepřímo závisí na celkem devíti modulech a má s~nimi 124~625 řádek kódu. Podporuje obě verze Pythonu. Na GitHubu má 270 hvězd a za poslední měsíc byl stažen více než desettisíckrát.

Projekt vznikl v~roce 2011 a od té doby vyšla již více než dvacítka verzí. V~době zkoumání byla nejnovější verze pouze několik týdnů stará, proto vývoj hodnotím jako aktivní. Na vývoji se podílelo více než šest desítek vývojářů, většina z~nich formou drobné úpravy, která bývá rychle přijata i od lidí mimo projekt a mimo Mozilla Services.

\protect\hyperlink{code:cornice}{V~ukázce kódu} najdete příklad použití Cornice. V~ukázce je definována služba, která umožňuje použít GET a POST na nějakou hodnotu \verb!/values/{value}!, kde \verb!value! reprezentuje ASCII název té hodnoty.

\begin{listing}[htbp]
\caption{{\label{code:cornice}Příklad použití z~dokumentace Cornice \autocite{cornicedoc}}}
\begin{minted}[bgcolor=codebg]{python}
from cornice import Service

values = Service(name='foo', path='/values/{value}',
                 description="Cornice Demo")

_VALUES = {}


@values.get()
def get_value(request):
    """Returns the value.
    """
    key = request.matchdict['value']
    return _VALUES.get(key)


@values.post()
def set_value(request):
    """Set the value.

    Returns *True* or *False*.
    """
    key = request.matchdict['value']
    try:
        # json_body is JSON-decoded variant of the request body
        _VALUES[key] = request.json_body
    except ValueError:
        return False
    return True
\end{minted}
\end{listing}

\subsection{HATEOAS}\label{hateoas}

Cornice nenabízí předem připravené mechanismy k~prolinkování zdrojů. Pokud chcete použít JSON API, HAL či další obdobné standardy, budete je muset dodržet a naimplementovat sami. Cornice nezískává žádný bod.

\subsection{Přístupová práva}\label{pux159uxedstupovuxe1-pruxe1va}

Cornice nenabízí žádné zabudované možnosti, jak řešit přístupová práva. Ve výchozím stavu je celé API přístupné všem, můžete však napsat vlastní funkci v~Pythonu, která práva bude řešit. Toto na jednu stranu nabízí téměř neomezené možnosti, na stranu druhou to není příliš pohodlné. Cornice získává jeden bod.

Cornice působí jako solidní nízkoúrovňový REST framework: Pokud víte, co děláte, můžete pomocí něj naimplementovat REST službu, ale neudělá příliš věcí za vás. Speciální funkcí je pak podpora SPORE\footnote{Specification to a POrtable Rest Environment \autocite{spore}} \autocite{cornicespore}.

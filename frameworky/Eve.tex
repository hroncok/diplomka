\section{Eve}\label{eve}

\begin{figure}
\centering
\includegraphics{pdfs/eve}
\caption{Logo Eve \autocite{evepic}\label{pic:eve}}
\end{figure}

Eve je open-source REST API framework navržený „pro lidi“. Umožňuje snadno vytvořit a nasadit vysoce upravitelné, plně funkční RESTful webové služby. Eve stojí nad nástroji Flask, Redis, Cerberus, Events a podporuje MongoDB i SQL backendy. \autocite{eve}

Eve vychází z~následujícího principu: Máte nějaká data a chcete k~nim vytvořit REST API, pokud možno co nejvíce automaticky. Prakticky bez práce nabízí mj. tyto funkce a možnosti \autocite{eveslides}:

\begin{itemize}
\tightlist
\item
  filtrování,
\item
  řazení,
\item
  stránkování,
\item
  projekce,
\item
  vnořené zdroje,
\item
  JSON nebo XML serializaci,
\item
  HATEOAS,
\item
  ukládání souborů,
\item
  limitování přístupu,
\item
  cache,
\item
  hromadné vkládání,
\item
  kontrolu integrity (pomocí ETagu),
\item
  validaci dat,
\item
  GeoJSON,
\item
  autentizaci a autorizaci,
\item
  podporu obou verzí Pythonu i PyPy,
\item
  verzování API,
\item
  generovanou dokumentaci.
\end{itemize}

Jak vidno, možností poskytuje opravdu mnoho. Příklad použití si můžete prohlédnout \protect\hyperlink{code:eve}{v~ukázce kódu}.

\begin{listing}[htbp]
\caption{{\label{code:eve}Příklad použití z~dokumentace Eve \autocite{evedoc}}}
\begin{minted}[bgcolor=codebg]{python}
# run.py
from eve import Eve
app = Eve()

if __name__ == '__main__':
    app.run()


# settings.py
DOMAIN = {'people': {}}


# GET /
{
    "_info": {
        "server": "Eve",
        "version": "a.b.c",
        "api_version": "x.y.z"
    },
    "_links": {
        "child": [
            {
                "href": "people",
                "title": "people"
            }
        ]
    }
}


# GET /people
{
    "_items": [],
    "_links": {
        "self": {
            "href": "people",
            "title": "people"
        },
        "parent": {
            "href": "/",
            "title": "home"
        }
    }
}
\end{minted}
\end{listing}

Projekt vznikl v~roce 2012, od té doby vyšlo dvacet verzí, poslední cca tři týdny před psaním tohoto textu. Jedná se tedy o~aktivní projekt. Stojí za ním jednotlivec, Nicola Iarocci, a přispělo do něj více než sto dalších přispěvatelů \autocite{evecontributors}. Eve je vydáno pod BSD licencí \autocite{BSD3}.

Eve závisí celkem na deseti modulech (včetně Flasku a Werkzeugu) a tyto moduly již nemají žádné další závislosti. Celkem se závislostmi má 35~009 řádků kódu. Závislost na Python modulech pro MongoDB bohužel není volitelná.

\subsection{HATEOAS}\label{hateoas}

Eve automaticky prolinkovává jednotlivé zdroje a drží se konceptu HATEOAS \autocite{evehateoas}. Tuto funkci není potřeba speciálně nastavovat ani implementovat, je implicitně zapnutá. Každá odpověď na metodu GET obsahuje položku \verb!_links! s~odkazy na rodiče, subsekce, předchozí a další stránky apod. Příklad můžete vidět \protect\hyperlink{code:evehateoas}{v~ukázce}.

Autoři pracují na přímé podpoře pro JSON-LD, HAL i Siren \autocite{eveslides}.

V~této oblasti Eve hodnotím třemi body.

\begin{listing}[htbp]
\caption{{\label{code:evehateoas}Příklad HATEOAS principu z~Eve \autocite{evehateoas}}}
\begin{minted}[bgcolor=codebg]{python}
{
    "_links": {
        "self": {
            "href": "people",
            "title": "people"
        },
        "parent": {
            "href": "/",
            "title": "home"
        },
        "next": {
            "href": "people?page=2",
            "title": "next page"
        },
        "last": {
            "href": "people?page=10",
            "title": "last page"
        }
    }
}
\end{minted}
\end{listing}

\subsection{Přístupová práva}\label{pux159uxedstupovuxe1-pruxe1va}

Eve umožňuje několik způsobů autentizace, například pomocí tokenu nebo HMAC\footnote{Hash Message Authentication Code \autocite{hmac}} \autocite{eveauth}. Pomocí externích knihoven je snadné přidat i OAuth 2 \autocite{eveoauth}.

Eve umožňuje nastavovat přístupová práva podle rolí pro celé API, nebo jen pro některé zdroje, stejně tak pro konkrétní HTTP metody \autocite{eveauth}.

Dávám tedy i zde Eve tři body.

Celkově se Eve jeví jako framework s~mnoha funkcemi, který dokáže ušetřit velké množství práce. Vytknul bych snad jen přílišnou vázanost na MongoDB, která je často patrná především z~dokumentace.

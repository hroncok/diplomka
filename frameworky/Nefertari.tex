\section{Nefertari}\label{nefertari}

Nefertari je REST API framework pro Pyramid, který používá Elasticsearch pro čtení a MongoDB nebo PostgreSQL pro zápis \autocite{nefertari}.

V~Nefertari je nejprve potřeba připravit model, což je entita mapovaná na databázi, a k~danému modelu vytvořit pohled, což je mapování dané entity na HTTP metody. Ukázkový model a pohled můžete vidět \protect\hyperlink{code:nefertarimodel}{v~ukázkách} \protect\hyperlink{code:nefertariview}{a}. Serializaci do JSONu a mapování na URL za vás obstará framework.

\begin{listing}[htbp]
\caption{{\label{code:nefertarimodel}Příklad použití z~dokumentace Nefertari (model) \autocite{nefertarimodel}}}
\begin{minted}[bgcolor=codebg]{python}
from datetime import datetime
from nefertari import engine as eng
from nefertari.engine import BaseDocument


class Story(BaseDocument):
    __tablename__ = 'stories'

    _auth_fields = [
        'id', 'updated_at', 'created_at', 'start_date',
        'due_date', 'name', 'description']
    _public_fields = ['id', 'start_date', 'due_date', 'name']

    id = eng.IdField(primary_key=True)
    updated_at = eng.DateTimeField(onupdate=datetime.utcnow)
    created_at = eng.DateTimeField(default=datetime.utcnow)

    start_date = eng.DateTimeField(default=datetime.utcnow)
    due_date = eng.DateTimeField()

    name = eng.StringField(required=True)
    description = eng.TextField()
\end{minted}
\end{listing}

\begin{listing}[htbp]
\caption{{\label{code:nefertariview}Příklad použití z~dokumentace Nefertari (pohled) \autocite{nefertariview}}}
\begin{minted}[bgcolor=codebg]{python}
from nefertari.view import BaseView
from example_api.models import Story


class StoriesView(BaseView):
    Model = Story

    def index(self):
        return self.get_collection_es()

    def show(self, **kwargs):
        return self.context

    def create(self):
        story = self.Model(**self._json_params)
        return story.save(self.request)

    def update(self, **kwargs):
        story = self.Model.get_item(
            id=kwargs.pop('story_id'), **kwargs)
        return story.update(self._json_params, self.request)

    def replace(self, **kwargs):
        return self.update(**kwargs)

    def delete(self, **kwargs):
        story = self.Model.get_item(
            id=kwargs.pop('story_id'), **kwargs)
        story.delete(self.request)

    def delete_many(self):
        es_stories = self.get_collection_es()
        stories = self.Model.filter_objects(es_stories)

        return self.Model._delete_many(stories, self.request)

    def update_many(self):
        es_stories = self.get_collection_es()
        stories = self.Model.filter_objects(es_stories)

        return self.Model._update_many(
            stories, self._json_params, self.request)
\end{minted}
\end{listing}

Nefertari závisí na devíti, nepřímo na osmnácti modulech, což je oproti jiným zkoumaným frameworkům opravdu hodně. Instalace obsahuje celkem 54~339 řádků kódu. Podporován je Python 2 i 3. Projekt je distribuován pod permisivní Apache licencí \autocite{apache}.

První commit v~projektu se datuje na březen 2015, jedná se tedy, v~době psaní tohoto textu, zhruba o~jeden rok starý projekt. Od té doby vyšlo téměř patnáct verzí, poslední v~listopadu 2015. Za projektem stojí startup Brandicted\footnote{Hlavní služba startupu Brandicted.com je v~době psaní tohoto textu nedostupná. Je otázkou, zdali jde o~náhodu, nebo má Nefertari nejistou budoucnost. Repozitář na GitHubu (který má 37 hvězd) se přesunul do organizace \emph{ramses-tech}, která ale obsahuje stejné vývojáře jako původní organizace \emph{brandicted}.}, kromě nich do projektu příliš mnoho lidí nepřispívá.

\subsection{HATEOAS}\label{hateoas}

Dokumentace Nefertari se nezmiňuje o~způsobu, jak jednotlivé zdroje prolinkovat. V~části \emph{Vize} \autocite{nefertarivision} dokonce přímo říká:

\begin{quote}
Pro nás znamená „REST API“ něco jako „HTTP metody namapované na CRUD\footnote{\emph{Create}, \emph{Retrieve}, \emph{Update}, \emph{Delete}} operace nad zdroji popsanými v~JSONu“. Nesnažíme se o~úplný HATEOAS, ani o~naplnění akademického ideálu o~RESTu.
\end{quote}

Nedostává tedy žádné body.

\subsection{Přístupová práva}\label{pux159uxedstupovuxe1-pruxe1va}

Nefertari používá model autentizace z~frameworku Pyramid, pomocí cookies \autocite{nefertariauth}, což je pro REST API nevyhovující. Přístupová práva oproti tomu umožňuje nastavit velice variabilně na úrovni jednotlivých operací a zdrojů \autocite{nefertariauth}. Dostává tedy jeden bod.

Nefertari vede k~tomu, že se vývojář o~některé věci vůbec nemusí starat: jak přesně jsou data uložena v~databázi nebo jak se mapují zdroje na URL. To může být velkou výhodou, ale i nevýhodou. Podle dokumentace se zdá, že jednotlivá výchozí chování nelze příliš ovlivnit. Nefertari jistě ušetří mnoho práce za cenu svobody volby. Absence HATEOAS a autentizace pomocí cookies můj pohled na Nefertari příliš nezlepší. Existují jistě situace, kde bude Nefertari velmi vhodná, ale není dostatečně flexibilní pro širokou škálu případů.

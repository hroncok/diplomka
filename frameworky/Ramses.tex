\section{Ramses (rozšíření pro Nefertari)}\label{ramses-rozux161uxedux159enuxed-pro-nefertari}

\begin{figure}
\centering
\includegraphics{pdfs/ramses}
\caption{Logo Ramsesu \autocite{ramses}\label{pic:ramses}}
\end{figure}

Ramses je framework, který generuje RESTful API pomocí RAMLu \footnote{\emph{RESTful API Modeling Language}, postavený na YAMLu \autocite{raml}}. Používá Pyramid a Nefertari \autocite{ramsesdoc}.

Ramses přináší stejné funkce jako Nefertari, o~které jsem psal \protect\hyperlink{nefertari}{v~části}. Místo kódu v~Pythonu se však používá deskriptivní jazyk RAML.

Ramses přímo závisí na sedmi modulech, nepřímo pak na téměř třiceti, instalace má celkem 68~594 řádků kódu. V~počtu závislostí je tak v~negativním slova smyslu vítězem.

Projekt vytváří stejní autoři jako Nefertari a všechny informace o~aktivitě jsou prakticky stejné. Projekt vznikl na přelomu února a března 2015, poslední z~dvanácti verzí vyšla v~listopadu téhož roku. Kód je distribuován pod permisivní Apache licencí \autocite{apache}, stejně jako Nefertari. Na rozdíl od Nefertari má na GitHubu více než dvě stovky hvězd.

Zkrácený příklad RAML souboru pro vytvoření API můžete vidět \protect\hyperlink{code:ramses}{v~ukázce}. V~odpovědi \protect\hyperlink{code:ramsesreply}{v~ukázce} pak můžete vidět absenci prolinkování.

\begin{listing}[htbp]
\caption{{\label{code:ramses}Příklad použití Ramsesu \autocite{ramsespizza}}}
\begin{minted}[bgcolor=codebg]{yaml}
#%RAML 0.8
---
title: pizza_factory API
documentation:
    - title: pizza_factory REST API
      content: |
        Welcome to the pizza_factory API.
baseUri: http://{host}:{port}/{version}
version: v1
mediaType: application/json
protocols: [HTTP]

/cheeses:
    displayName: Collection of different cheeses
    get:
        description: Get all cheeses
    post:
        description: Create a new cheese
        body:
            application/json:
                schema: !include schemas/cheeses.json

    /{id}:
        displayName: A~particular cheese ingredient
        get:
            description: Get a particular cheese
        delete:
            description: Delete a particular cheese
        patch:
            description: Update a particular cheese

/pizzas:
    displayName: Collection of pizza styles
    get:
        description: Get all pizza styles
    post:
        description: Create a new pizza style
        body:
            application/json:
                schema: !include schemas/pizzas.json

    /{id}:
        displayName: A~particular pizza style
        get:
            description: Get a particular pizza style
        delete:
            description: Delete a particular pizza style
        patch:
            description: Update a particular pizza style

# ...
\end{minted}
\end{listing}

\begin{listing}[htbp]
\caption{{\label{code:ramsesreply}Odpověď Ramsesu \autocite{ramsespizza}}}
\begin{minted}[bgcolor=codebg]{python}
# POST /api/pizzas name=hawaiian toppings:=[1,2] ...
{
    "data": {
        "_type": "Pizza",
        "_version": 0,
        "cheeses": [
            1
        ],
        "crust": 1,
        "crust_id": 1,
        "description": null,
        "id": 1,
        "name": "hawaiian",
        "sauce": 1,
        "sauce_id": 1,
        "self": "http://localhost:6543/api/pizzas/1",
        "toppings": [
            1,
            2
        ],
        "updated_at": null
    },
    "explanation": "",
    "id": "1",
    "message": null,
    "status_code": 201,
    "timestamp": "2015-06-05T18:47:53Z",
    "title": "Created"
}
\end{minted}
\end{listing}

Vzhledem k~tomu, že Ramses je vrstva abstrakce nad Nefertari, nebudu zde opakovat sekce o~HATEOAS a přístupových právech, jelikož by byly prakticky stejné. Vytváření REST API pomocí RAML souborů je možná směr, kterým se v~budoucnu lidstvo vydá, ale bojím se, že na Ramsesu je třeba ještě zapracovat. Otázkou je, jestli bude dále vyvíjen.

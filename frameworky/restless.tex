\section{restless}\label{restless}

Restless je miniframework pro tvorbu REST API. Podporuje Django, Flask, Pyramid, Tornado a Itty\footnote{Itty je webový framework od autora restlessu.}, ale měl by fungovat s~jakýmkoliv webovým frameworkem v~Pythonu \autocite{restless}.

Hlavní myšlenkou restlessu je udělat věci jednoduše a příliš je nekomplikovat, mezi hlavní výhody patří \autocite{restless}:

\begin{itemize}
\tightlist
\item
  malý a rychlý kód,
\item
  výchozí výstup v~JSONu,
\item
  koncept RESTful,
\item
  podpora Pythonu 3.3+ (i staršího 2.6+),
\item
  flexibilita.
\end{itemize}

Restless v~dokumentaci rovnou uvádí, že nebude podporovat automatickou integraci ORM, XML, autorizaci ani HATEOAS \autocite{restless}\autocite{restlessp}. Dostává tedy u~obou bodovaných kritérií nula bodů.

Za projektem stojí jednotlivec Daniel Lindsley, který je mj. autorem frameworku Tastypie, o~kterém budu mluvit \protect\hyperlink{tastypie}{v~části}. Dokumentace restlessu projekt Tastypie často zmiňuje a zdůrazňuje, že restless vznikl poučením se z~chyb při tvorbě Tastypie. Hlavní chybou bylo pokoušet se o~vytvoření příliš „všemocného“ frameworku, restless jde tedy opačnou cestou a většinu rozhodnutí nechává na uživateli \autocite{restlessp}.

Restless, na rozdíl od Tastypie, není vázán přímo na Django, ale webový framework si můžete zvolit. Přímo v~kódu existují třídy pro frameworky Django, Flask, Pyramid, Tornado a Itty, od kterých stačí dědit. Pro jiný framework si takovou třídu můžete samozřejmě dopsat sami. V~případě změny frameworku by mělo stačit třídu vyměnit. Příklad z~dokumentace s~použitím třídy pro Django můžete vidět \protect\hyperlink{code:restless}{v~ukázce}.

Restless závisí pouze na knihovně six, kvůli zpětné kompatibilitě s~Pythonem~2. Instalace tak zabírá pouze čtvrt mebibajtu a obsahuje 1~140 řádek kódu, ale tato informace je zavádějící, protože restless ještě vyžaduje nějaký webový framework, samostatně nefunguje.

\begin{listing}[htbp]
\caption{{\label{code:restless}Příklad použití s~Djangem z~dokumentace restlessu \autocite{restlessgh}}}
\begin{minted}[bgcolor=codebg]{python}
from django.contrib.auth.models import User

from restless.dj import DjangoResource
from restless.preparers import FieldsPreparer

from posts.models import Post


class PostResource(DjangoResource):
    # Controls what data is included in the serialized output.
    preparer = FieldsPreparer(fields={
        'id': 'id',
        'title': 'title',
        'author': 'user.username',
        'body': 'content',
        'posted_on': 'posted_on',
    })

    # GET /
    def list(self):
        return Post.objects.all()

    # GET /pk/
    def detail(self, pk):
        return Post.objects.get(id=pk)

    # POST /
    def create(self):
        return Post.objects.create(
            title=self.data['title'],
            user=User.objects.get(username=self.data['author']),
            content=self.data['body']
        )

    # PUT /pk/
    def update(self, pk):
        try:
            post = Post.objects.get(id=pk)
        except Post.DoesNotExist:
            post = Post()

        post.title = self.data['title']
        post.user = User.objects.get(username=self.data['author'])
        post.content = self.data['body']
        post.save()
        return post

    # DELETE /pk/
    def delete(self, pk):
        Post.objects.get(id=pk).delete()
\end{minted}
\end{listing}

Projekt vznikl v~lednu 2014, autor jej aktivně vyvíjel do srpna toho roku, od té doby prakticky pouze přijímá cizí příspěvky, kterých však není příliš mnoho, poslední byl přijat v~létě 2015. Od té doby se kupí další a další, čekající na schválení, které možná nikdy nepřijde. Zatím poslední verze, v~pořadí sedmá, vyšla v~srpnu 2014. Daniel Lindsley se restlessu zjevně nevěnuje. Uživatelé však nadále hlásí chyby a snaží se přispět svým kódem. Více než pět stovek hvězd na GitHubu u~projektu, který byl aktivně vyvíjen půl roku, svědčí o~tom, že měl potenciál.

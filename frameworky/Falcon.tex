\section{Falcon}\label{falcon}

Falcon je neuvěřitelně rychlý, minimalistický Python webový framework pro tvorbu „cloudových API“ a aplikačních backendů \autocite{falcon}. Mezi hlavní přednosti podle webové stránky \autocite{falcon} patří:

\begin{itemize}
\tightlist
\item
  závislost pouze na modulech \verb!six! a \verb!mimeparse!,
\item
  rychlejší zpracování požadavků než u~jiných populárních frameworků,
\item
  podpora WSGI, CPythonu 2.6, 2.7, 3.3 a 3.4 i PyPy,
\item
  svoboda volby detailů,
\item
  spolehlivost.
\end{itemize}

\begin{figure}
\centering
\includegraphics{images/falcon}
\caption{Logo Falconu \autocite{falconpic}\label{pic:falcon}}
\end{figure}

Falcon je bezesporu minimalistický -- společně se závislostmi má pouze 3~034 řádků kódu. Je šířen pod permisivní Apache licencí \autocite{apache} a nevyžaduje žádný webový framework.

Příklad použití můžete najít \protect\hyperlink{code:falcon}{v~ukázce}. Jak je vidět, pomocí Falconu jdou vytvářet REST API, ale jedná se o~velmi nízkoúrovňový framework, který spíše zastává vrstvu mezi HTTP a aplikací než velkého pomocníka při tvorbě API.

\begin{listing}[htbp]
\caption{{\label{code:falcon}Příklad použití z~webu Falconu \autocite{falcon}}}
\begin{minted}[bgcolor=codebg]{python}
# sample.py
import falcon
import json

class QuoteResource:
    def on_get(self, req, resp):
        """Handles GET requests"""
        quote = {
            'quote': 'I\'ve always been more interested '
                     'in the future than in the past.',
            'author': 'Grace Hopper'
        }

        resp.body = json.dumps(quote)

api = falcon.API()
api.add_route('/quote', QuoteResource())
\end{minted}
\end{listing}

Projekt vytváří firma Rackspace pod vedením Kurta Griffithse. Do projektu přispívají i jednotlivci mimo Rackspace. Vznikl v~roce 2012 a od té doby vyšlo celkem 27 verzí. Dva týdny před psaním tohoto textu vyšla verze 1.0.0rc1, brzy se tedy můžeme těšit na verzi 1.0.0. Jedná se o~aktivní projekt, který se může chlubit stoprocentním pokrytím testy \autocite{falconcoverage}.

Vzhledem k~nízkoúrovnosti frameworku neexistují žádné automatické mechanismy pro správu přístupových práv či HATEOAS. Falcon tedy za oba aspekty získává nula bodů.

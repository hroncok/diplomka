\section{ripozo}\label{ripozo}

\begin{figure}
\centering
\includegraphics{images/ripozo}
\caption{Logo ripoza \autocite{ripozopic}\label{pic:ripozo}}
\end{figure}

Ripozo je nástroj pro vytváření RESTful/HATEOAS API. Poskytuje silné, jednoduché, plně kvalifikované odkazy mezi zdroji; podporuje více protokolů (Siren a HAL). Ripozo je velmi flexibilní, dá se použít s~libovolným webovým frameworkem v~Pythonu a libovolnou databází. \autocite{ripozo}

Základní příklad použití typu \emph{hello world} můžete vidět \protect\hyperlink{code:ripozo}{v~ukázce}. V~ukázce je vynecháno napojení na webový framework. Můžete využít existujících knihoven pro napojení na Django a Flask, či si napsat vlastní třídu pro napojení na jiný webový framework \autocite{ripozo}. Příklad, který přes REST API nabízí kompletní CRUD+L\footnote{\emph{Create}, \emph{Retrieve}, \emph{Update}, \emph{Delete} a \emph{List} \autocite{crud}}, pak můžete vidět \protect\hyperlink{code:ripozocrudl}{v~ukázce}. Pokud chcete nabízet jen některé akce, můžete použít mixiny\footnote{Mixin je třída, kterou v~Pythonu použijete jako rodiče nebo jednoho z~rodičů, abyste rozšířili funkcionalitu. Ničím se neliší od jiné třídy, termín mixin se používá pouze na odlišení významu. V~tomto konkrétním případě tak například můžete použít mixiny \emph{restmixins.Create} a \emph{restmixins.List} pro poskytnutí akcí pouze pro čtení. \autocite{mixin}}.

\begin{listing}[htbp]
\caption{{\label{code:ripozo}Příklad použití z~dokumentace ripoza \autocite{ripozo}}}
\begin{minted}[bgcolor=codebg]{python}
from ripozo import apimethod, adapters, ResourceBase
# import the dispatcher class for your preferred webframework

class MyResource(ResourceBase):
    @apimethod(methods=['GET'])
    def say_hello(cls, request):
        return cls(properties=dict(hello='world'))

# initialize the dispatcher for your framework
# e.g. dispatcher = FlaskDispatcher(app)
dispatcher.register_adapters(adapters.SirenAdapter,
                             adapters.HalAdapter)
dispatcher.register_resources(MyResource)
\end{minted}
\end{listing}

\begin{listing}[htbp]
\caption{{\label{code:ripozocrudl}Příklad použití z~dokumentace ripoza (CRUD+L) \autocite{ripozo}}}
\begin{minted}[bgcolor=codebg]{python}
from ripozo import restmixins
from fake_ripozo_extension import Manager
# An ORM model for example a sqlalchemy or Django model:
from myapp.models import MyModel

class MyManager(Manager):
    fields = ('id', 'field1', 'field2',)
    model = MyModel

class MyResource(restmixins.CRUDL):
    manager = MyManager()
    pks = ('id',)

# Create your dispatcher and register the resource...
\end{minted}
\end{listing}

Projekt vznikl v~roce 2014, od té doby vyšlo více než třicet verzí, nejnovější asi měsíc před psaním tohoto textu. Za projektem stojí firma Vertical Knowledge, vyvíjí jej hlavně Tim Martin, ale přispěli i jednotlivci nesouvisející s~touto firmou. Ripozo je distribuováno pod copyleftovou licencí GNU General Public License verze 2 \autocite{GPLv2} nebo vyšší, čímž se odlišuje od naprosté většiny ostatních zde diskutovaných frameworků.

Instalace závisí jen na knihovně six, kvůli kompatibilitě s~oběma verzemi Pythonu, zabírá pouze půl mebibajtu a obsahuje 2~130 řádků kódu. Vzhledem k~tomu, že instalace samotného ripoza je nepoužitelná, jelikož je potřeba použít nějaký webový framework, je tato informace zavádějící. Například po instalaci modulů na spolupráci s~Flaskem a SQLAlchemy je již závislostí sedm (nepočítaje tři vlastní moduly \verb!ripozo!, \verb!flask-ripozo! a \verb!ripozo-sqlalchemy!) a instalace zabírá 14 MiB.

\subsection{HATEOAS}\label{hateoas}

Již v~úvodu jsem zmínil, že ripozo umožňuje jednoduše vytvářet linky mezi zdroji ve stylu HATEOAS a také že ripozo podporuje Siren \autocite{siren} a HAL \autocite{hal}. Získává tedy tři body. Představu o~vytváření odkazů získáte nejlépe z~\protect\hyperlink{code:ripozolink}{ukázky}.

\begin{listing}[htbp]
\caption{{\label{code:ripozolink}Příklad použití z~dokumentace ripoza (linkování) \autocite{ripozo}}}
\begin{minted}[bgcolor=codebg]{python}
from ripozo import restmixins, Relationship

class MyResource(restmixins.CRUDL):
    manager = MyManager()
    pks = ('id',)
    _relationships = [Relationship('related',
                                   relation='RelatedResource')]

class RelatedResource(restmixins.CRUDL)
    manager = RelatedManager()
    pks = ('id',)
\end{minted}
\end{listing}

\subsection{Přístupová práva}\label{pux159uxedstupovuxe1-pruxe1va}

Ripozo nenabízí přímo žádnou funkcionalitu pro autentizaci či autorizaci. Obsahuje však možnost předzpracovávat požadavky pomocí funkcí. V~dokumentaci se říká, že tímto způsobem můžete například zpřístupnit zdroj pouze autentizovaným uživatelům \autocite{ripozoprepost}. Ripozo zde tedy dostává dva body.

Ripozo je framework, který umožňuje vytvářet RESTful HATEOS API pomocí Siren a HAL, prakticky bez práce. Možnost výběru vlastního frameworku i databáze je velké plus. Nevýhodou může v~některých případech být copyleftová licence\footnote{Richard M. Stallman by určitě namítal, že to je výhodou.}.

\section{Flask-RESTful}\label{flask-restful}

\begin{figure}
\centering
\includegraphics{images/flask-restful}
\caption{Logo Flask-RESTful \autocite{flaskresfulpic}\label{pic:flaskresful}}
\end{figure}

Flask-RESTful je rozšíření Flasku, které přidává podporu pro rychlé vytváření RESTových API. Jedná se o~tenkou vrstvu abstrakce, která by měla fungovat s~existujícím ORM a dalšími knihovnami. Flask-RESTful je navržen tak, aby ho uživatelé obeznámení s~Flaskem co nejrychleji pochopili. \autocite{flaskresful}

Za vývojem Flask-RESTful stojí firma Twilio, ale přispělo do něj více než sto jednotlivců. Je zveřejněn pod BSD licencí \autocite{BSD3}. Závisí na Flasku a dalších třech modulech. Celkově tak nepřímo závisí na modulech devíti a společně s~nimi má 27~718 řádků kódu. Na GitHubu má necelé dva tisíce hvězd a na PyPI má za poslední měsíc více než 170 tisíc stažení. Podporuje obě verze Pythonu. Projekt vznikl v~roce 2012, od té doby vyšlo 27 verzí, poslední v~prosinci roku 2015.

Příklad použití můžete vidět \protect\hyperlink{code:flaskresful}{v~ukázce}\footnote{Příklad byl mírně zhuštěn za účelem lepší prezentace na straně formátu A4.}.

\begin{listing}[htbp]
\caption{{\label{code:flaskresful}Příklad použití z~dokumentace Flask-RESTful \autocite{flaskrestfuldoc}}}
\begin{minted}[bgcolor=codebg]{python}
from flask import Flask
from flask_restful import reqparse, abort, Api, Resource

app = Flask(__name__)
api = Api(app)

TODOS = {'todo1': {'task': 'build an API'}, ...}

def abort_if_todo_doesnt_exist(todo_id):
    if todo_id not in TODOS:
        abort(404, message="Todo {} doesn't exist".format(todo_id))

parser = reqparse.RequestParser()
parser.add_argument('task')

# shows a single todo item and lets you delete a todo item
class Todo(Resource):
    def get(self, todo_id):
        abort_if_todo_doesnt_exist(todo_id)
        return TODOS[todo_id]

    def delete(self, todo_id):
        abort_if_todo_doesnt_exist(todo_id)
        del TODOS[todo_id]
        return '', 204

    def put(self, todo_id):
        args = parser.parse_args()
        task = {'task': args['task']}
        TODOS[todo_id] = task
        return task, 201

# shows a list of all todos, and lets you POST to add new tasks
class TodoList(Resource):
    def get(self):
        return TODOS

    def post(self):
        args = parser.parse_args()
        todo_id = int(max(TODOS.keys()).lstrip('todo')) + 1
        todo_id = 'todo%i' % todo_id
        TODOS[todo_id] = {'task': args['task']}
        return TODOS[todo_id], 201

# Actually setup the Api resource routing here
api.add_resource(TodoList, '/todos')
api.add_resource(Todo, '/todos/<todo_id>')

if __name__ == '__main__':
    app.run(debug=True)
\end{minted}
\end{listing}

Flask-RESTful je nízkoúrovňový framework, který zjednodušuje tvorbu REST API oproti použití čistého Flasku, ale nepřináší žádné pokročilé funkce jako podporu autentizace a autorizace, či prolinkování a HATEOAS. Nedostává tedy žádné body. Ze zajímavých funkcí Flask-RESTful mohu jmenovat vyjednávání o~obsahu či podporu \emph{blueprintů} (koncept z~Flasku \autocite{blueprint}).
